%*******************************************************************************
% Example chapter file for books, Copyright A K Peters, Ltd.
%*******************************************************************************
\chapter{Exploring Mobile vs. Desktop OpenGL Performance}{Jon McCaffrey}
 \label{Exploring-Mobile-vs-Desktop-OpenGL-Performance}

\section{Introduction}

The stunning rise of mobile platforms has created a large nascent market for new 3D applications and games where, excitingly, OpenGL ES is the \textit{lingua franca} for graphics. However, mobile platforms and GPUs have performance profiles and characteristics that may be unfamiliar and surprising to desktop developers.  Developers and applications hoping to make the transition

% TODO Finish introduction

\section{Constraint Inspires Creativity}\label{YourName:Constraints-Inspire-Creativity}

\subsection{Differences in Scale}\label{Jon-McCaffrey:Architectural-Differences}
Modern mobile devices are capable and remarkable devices.  However, they face much greater limitations than desktop systems in terms of cost, die size, power consumption, and heat dissipation.

TODO better scale context

\subsection{Differnces in Rendering Architecture}\label{Jon-McCaffrey:differences-in-rendering-architecture}

Mobile and desktop GPUs don't differ only in scale.  Many mobile GPUs such as the PowerVR SGX543MP2 used in the Apple iPhone 4S/iPad 2 and the Mali-400 used in the Samsung Galaxy S2 use a rendering architecture known as \textit{Tile-Based Deferred Rendering (TBDR)}.  In contrast, familiar desktop GPUs from Nvidia and ATI and mobile GPUs like the GeForce ULV GPU use in the Samsung Galaxy Tab 10.1 use \textit{Immediate Mode Rendering}.

TODO link for TBDR vs IMR http://www.anandtech.com/Show/Index/4686?cPage=13&all=False&sort=0&page=15&slug=samsung-galaxy-s-2-international-review-the-best-redefined

In IMRs, vertices are transformed once and each primitive is rasterized essentially in order.  If a fragment passes depth-testing (assuming the platform has early-z), it will be shaded and will write to the framebuffer.  However, a later fragment may over-write this pixel, nullifying the earlier work done and writing the framebuffer again.  This issue is known as overdraw.  Even without overdraw, the depth buffer still must be read in order to reject fragments.

TBDRs instead divide the scene into tiles.  All draw commands are buffered, or deferred.  At the end of the frame, for each tile, all geometry for the scene is re-rendered and rasterized with that tile scissored out.  Before any fragments are shaded, all \textit{Hidden Surface Removal (HSR)} and depth-testing is performed in a fast framebuffer cache.  Assuming opaque geometry, each pixel is then shaded and writted to the framebuffer exactly once.

These architectures have different strengths and weaknesses we will consider, but it's important to note that the mobile GPU landscape is not homogenous.  Optimizations may benefit the different architectures very differently.

\subsection{Differences in Memory Architecture}\label{Jon-McCaffrey:differences-in-memory-architecture}

On desktop systems, high-end GPUs are discrete devices which communicate with the rest of the system via a peripheral bus like \textit{PCIe}.  For good performance, this means that GPUs must include their own dedicated memory.  While this increases cost, it is an optimization opportunity since this memory can be optimized for graphics workloads.

For example, the Nvidia \textit{Fermi} architecture uses GDDR5 memory that is heavily partitioned (TODO link http://www.techspot.com/review/263-nvidia-geforce-gtx-480/page2.html) to allow for very wide memory interface.  There is also no competition for this bandwidth, since except for uploads from the rest of the system and scan-out for output devices, the GPU is the only user of this memory.

In mobile devices, on the other hand, the GPU is usually integrated into the same \textit{System-on-a-Chip (SoC)} as the CPU and other components, and to save cost, power, size, and complexity, shares the same RAM and memory interface.  This is known as an \textit{Unified Memory Architecture (UMA)}.  A common memory type might be the low-power LPDDR2, which has a 32-bit wide interface. (http://www.anandtech.com/show/4144/lg-optimus-2x-nvidia-tegra-2-review-the-first-dual-core-smartphone/5).  Not only is this memory general-purpose, the GPU now shares bandwidth with other parts of the system like the CPU, network, camera, multimedia, and display.   

There are some advantages to an unified memory architecture.  With discrete GPU's the peripheral bus could become a bottleneck for transfers, especially with non-PCIe buses with asymmetric speeds (TODO link phttp://developer.download.nvidia.com/assets/gamedev/docs/Fast_Texture_Transfers.pdf?display=style-table).  With a UMA, OpenGL client and server data are in fact stored in the same RAM.  Even if it is not possible to directly access server data with \texttt{glMapBufferOES}, there are less performance cliffs lurking in transfers between OpenGL client and server data.

One limitation is that OpenGL ES does not yet have an extension for \textit{Pixel Buffer Objects (PBO)}, meaning that pixel data must be transferred synchronously.  This makes the comparatively cheap bandwidth between client and server data less useful.

\section{Reducing Memory Bandwidth}\label{Jon-McCaffrey:differences-in-memory-architecture}

Memory bandwidth pressure is one of the major performance bottlenecks on mobile devices.

TODO cite CPU/GPU read and write bandwidth numbers

\subsection{Relative Display Sizes}\label{Jon-McCaffrey:differences-in-memory-architecture}

Despite the tight power and cost constraints for mobile devices, the display sizes of modern mobile devices are a considerable fraction of the size of desktop displays.  

\begin{table}[htb]\centering
\begin{tabular}{|c|c|c|c|}
\hline%
\small{Device} & \small{Resolution} & \small{Percentage of 1280x1024 (20 in)} & \small{Percentage of 1920x1080 (24 in)}  \\
\hline%
\small{Motorola XOOM} & \small{1280x800} & \small{\%78.13} & \small{\%49.38}\\
\hline%
\small{Apple iPad 2} & \small{1024x768} & \small{\%58.63} & \small{\%37.06}\\
\hline%
\small{Apple iPhone 4S} & \small{960x640} & \small{\%46.89} & \small{\%29.63}\\
\hline%
\small{Samsung Galaxy S2} & \small{800x480} & \small{\%30.00} & \small{\%18.52}\\
\hline%
\end{tabular}
 \caption{Resolution comparison of desktop and mobile displays}
 \label{JonMcCaffrey:resolutions}
\end{table}

With the limited fragment compute throughput and memory bandwidth of mobile devices, these comparatively large display sizes mean that fragment shading and full-screen (or large-quad) operations can easily become a bottleneck.

There is also a large spread of resolution sizes within mobile devices, so testing on multiple devices is important, for performance testing as well as useabilty concerns.

\subsection{Framebuffer Bandwidth}

How much bandwidth is consumed on framebuffer operations anyways?  Assume the common configuration of 16-bit color with 16-bit depth, with a reasonable 1024x768 display.  Accessing every pixel in the framebuffer 60 times a second takes 94MB/s of bandwidth.  So simply to write all the pixels in a scene, with \%0 overdraw, takes 94MB/s of bandwidth.

TODO link http://developer.android.com/reference/android/opengl/GLSurfaceView.html

However, assuming an IMR architecture, to be able to render a scene, we also have to perform a depth-buffer read for each rendered pixel.  Both the depth-buffer and the color buffer are also usually cleared each frame.  

However, the memory bandwidth consumption of the final framebuffer doesn't end when the application is done writing it.  After \texttt{eglSwapBuffers}, it may need to be composited by the platform-specific windowing system, and then scanned out to the display.  Unlike on desktop systems which often have dedicated graphics or framebuffer memory, this will also consume system memory bandwidth.  This will consume an additional 96MB/s of bandwidth just for scanout, or at least an additional 288MB/s with composition(read, write, and scan-out).


Thus basic clear-fill-and-post operation consumes 470-658MB/s of bandwidth, so even basic operations consume a significant amount of memory bandwidth.  To put this in context, the iPad2 has been benchmarked at 2.3GB/s for stdlib writes.  http://www.anandtech.com/show/4215/apple-ipad-2-benchmarked-dualcore-cortex-a9-powervr-sgx-543mp2/2
If a 32-bit framebuffer is used, this number will be even greater (http://www.curious-creature.org/2010/12/04/gingerbread-and-32-bits-windows/).  

#TODO Bandwidth nums to put this in context

Applications using a 32-bit framebuffer that may be bandwidth-bound should experiment with a lower-precision format.  Since the output framebuffer is not often used in subsequent calculations, the loss of numerical precision is not propagated and magnified.  One valid concern is banding or quantization of smooth gradients.  However, this may be more of an issue in graphics, photography and compositing applications rather than games and 3d-applications, just because of the nature of the produced content.

\subsection{Texture Bandwidth}
Since texture accesses are often performed at least once per-pixel, these can be another large source of bandwidth consumption.

One simple way to reduce bandwidth is to lower the texture resolution.  Fewer texels, besides a smaller memory footprint, means better texture cache utilization and more efficient filtering.  The framebuffer resolution usually can't be lowered, since native resolution is expected.  Texture sizes are more flexible.  If assets have been ported from desktop, there may be room for optimization here.

For static textures (as opposed to textures resulting from off-screen render
targets) texture compression is another great way to save bandwidth,
loading-time, memory footprint, and disk space.  Even though work must be
done to de-compress the texture data when it is used, the smaller size of
compressed textures makes them friendlier to texture caching and memory
bandwidth, increasing run-time performance. 

One complication is that there are multiple incompatible formats for texture
compression supported via OpenGL ES2 extensions.  Example formats would be S3TC, available on Nvidia Tegra, and PVRTC, available on ImaginationTech PowerVR.

To support texture compression on multiple devices, your application must
either package multiple versions of their assets and dynamically choose the
correct ones, or perform the compression at run-time/install-time.  Performing
the compression at run or install-time must be done carefully to not slow down
your application, and gives up many of the benefits of improved loading-time
and disk-space, as well as the internet bandwidth required for
installation/download of the application on mobile devices.  S3TC has
compressions ratios between 4:1 and 8:1.
http://developer.motorola.com/docstools/library/understanding-texture-compression/
http://www.oldunreal.com/editing/s3tc/ARB_texture_compression.pdf

\Section{Reducing Fragment Workload}

Often related to memory bandwidth and resolution is fragment shading.  Besides trimming shader instructions and heavy operations, fragment shading can be optimized in other ways.

\subsection{Overdraw and Blending}
Overdraw is when pixels that have previously been shaded are over-written by later fragments in a scene.  On non-tiled architectures, overdraw wastes fragment processing, since the previous computed pixel value is over-written and lost.  The additional framebuffer writes also waste bandwidth.

On non-tiled architectures, this extra bandwidth consumption can limited by
sorting and rendering geometry from front to back.  This is especially
practical for static geometry which can be processed into a spatial data
structure during export.  An additional heuristic for games is to render the
player character first and the sky-box last.  For batches where front-to-back
object sorting is not practical, for example with complicated, interlocking
geometry or heavy use of alpha-testing, a depth-prepass can be used to
eliminate redundant pixel calculations, at the cost of repeated vertex shading
work, primitive assembly, and depth-buffer access.

TODO cite unity talk

Even without overdraw, heavy amounts of over-lapping geometry can still be expensive because of the depth buffer reads needed to reject pixels.  Primitive assembly, rasterization, and the pixel reject rate can also become limiting for large areas like sky-boxes.

TODO cite unity talk again.

One type of effect that can be particular expensive in terms of fragment
shading and read and write bandwidth is particle effects rendered via multiple
overlapping quads with blending.  These often overlap and are blended multiple
layers deep.  Each layer of overlap requires a read and write of the existing
framebuffer value, and an additional fragment computation and blending
operation.  One simple fix is tuning the size, count and style of particles.

\subsection {Full-Screen Effects} 
Full-screen post-processing effects are a major tool for visual effects in
modern games and graphics applications, and have been an area of innovation in
recent years.   Common applications of full-screen post-processing in games are
motion-blur, depth-of-field, screen-space ambient occlusion, light bloom, color
filtering, and tone-mapping.  Other applications such as photo-editing tools
may use full-screen or large-area effects for composition, blending, distortion
and filtering.

While it allows for beautiful and creative effects, full-screen post-processing
is an easy way to consume large amounts of bandwidth and fragment processing.
Such effects should be carefully weighed for their worth, and are prime
candidates for optimization.  

A full-screen pass implies at least a read and write of the framebuffer at full
resolution, which at 16-bit color and a 1024x768 resolution means 188MB/s
bandwidth.  One way to optimize these effects is to remove the extra
full-screen pass.  Some post-processing effects such as color-filtering or
tone-mapping that don't require knowledge of neighboring pixels or feedback
from rendering may be merged into the fragment shaders for the objects
themselves.  This may require the use of \textit{uber-shaders} or shader
generation, to allow for natural editing of object fragment shaders while
appending post-processing effects.

If the additional pass cannot be eliminated, then all full-screen post effects should be merged into a single pass.  This improves locality and saves redundant round-trips to framebuffer memory.

#TODO numbers here

One limitation of OpenGL ES 2.0 is the poor support for multiple render
targets.  This makes deferred shading impractical, since running a full pass of
the scene for each geomtry buffer is too expensive, but only 1 geometry buffer
can be rendered at a time.

\subsection{Off-screen pass}

Similar to full-screen effects are effects requiring off-screen render targets like environmental reflections, depth-map shadows, and light bloom.

One way to optimize full-screen effects that require a blurred image is to take
advantage of texture filtering hardware.  Rather than rendering a large
offscreen image, then taking multiple samples in a fragment shader to blur, the
scene can be rendered into a low-resolution offscreen target.  

The main fragment shader for the scene can then bind that target as a texture
and read from it with an appropriate texture-filtering mode such as GL_LINEAR.
The smaller size of the offscreen target makes this strategy particularly
cache-friendly.  This works well for light bloom and environmental reflection,
for example.  Even when the blurring due to texture filtering is not
beneficial, reducing off-screen target resolution may be an easy way to reduce
the fragment workload and memory bandwidth without a serious visual impact.

Whenever moving additional computations into the fragment shader of objects in
the scene, it is important on non-tiled architecture to minimize over-draw to
avoid wasted work.  One advantage of full-screen post-processing is that each
pixel is computed exactly once.

\subsection{Shaving Fragment Work}

#TODO light vs normal fragment shaders with approximately same work

\subsection{Vertex vs. Fragment work}

For lighting and shading the primary objects in your scene, IMR wisdom states that moving computations like lighting, specularity, and normalization from per-fragment to per-vertex can save performance at the cost of image quality, and this is still very true for IMR.

#TODO linkhttp://developer.apple.com/library/ios/#documentation/3DDrawing/Conceptual/OpenGLES_ProgrammingGuide/TechniquesforWorkingwithVertexData/TechniquesforWorkingwithVertexData.html#//apple_ref/doc/uid/TP40008793-CH107-SW1

However, for TBDRs, this performance wisdom is more dubious.  This is because mobile devices do have smaller screens that desktop, and because TBDRs must perform all vertex computations for each tile.  This means that heavy vertex shaders, even if they save fragment work, may be a performance drag on TBDR.  This is particularly true since TBDRs due little-to-no redundant fragment work.

Vertex processing may also be more of a bandwidth drain on TBDRs, since the attributes must be pulled again for each tile.  To lower this bandwidth, a lower-precision buffer format may be used such as \textit{OES_vertex_half_float}.  Interleaved vertex data, which interleaves the attributes for each vertex in the same buffer, is also much more efficient to fetch, since an entire vertex can be fetched in one linear read.  If there is a pre-transform attribute cache, this will also use it more efficiently.  
http://www.khronos.org/registry/gles/extensions/OES/OES_vertex_half_float.txt
\section{Conclusion}
http://aras-p.info/blog/2011/02/01/ios-shader-tricks-or-its-2001-all-over-again/
TODO conclude


\bibliographystyle{akpbib}
\bibliography{YourBib}











